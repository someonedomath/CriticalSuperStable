% * The preamble
% ! This TeX file uses 'amsart' for its document class. 
\documentclass[12pt,a4paper]{amsart}
\setlength{\textwidth}{\paperwidth}
\addtolength{\textwidth}{-2in}
\calclayout
% ! 'amsmath', 'amsthm', 'amsfonts' package is automatically loaded by 'amsart' class.
% ! Below is the configuration of 'amsmath' package.
\numberwithin{equation}{section}
\allowdisplaybreaks
% ! Below is the configuration of 'amsthm' package.
\theoremstyle{plain}
\newtheorem{thm}{Theorem}[section]
\newtheorem{lem}[thm]{Lemma}
\newtheorem{prop}[thm]{Proposition}
\newtheorem{cor}[thm]{Corollaray}
\newtheorem{conj}[thm]{Conjecture}
\theoremstyle{definition}
\newtheorem{defi}[thm]{Definition}
\newtheorem{exa}[thm]{Example}
\newtheorem{asp}{Assumption}
\newtheorem{iss}{Issue}
\newtheorem{rem}[thm]{Remark}
\newtheorem*{ack}{Acknowledgment}
% ! 'inputenc' package is used to specify the set of characters allowed to input into this TeX file.
% \usepackage[utf8]{inputenc}
% ! 'foutenc' package is used to specify the set of characters allowed to output into the generated pdf file.
% \usepackage[T1]{fontenc}
% ! 'amssymb', 'mathrsfs' and 'mathtools' packages provide additional math symbols from the AMS default symbol fonts. They are compatible with the AMS class and is recommended to be used.
\usepackage{amssymb}
\usepackage{mathtools}
\mathtoolsset{showonlyrefs}
\usepackage{mathrsfs}
% ! 'hyperref' package is used to handle cross-referencing in the generated pdf file.
\usepackage[backref]{hyperref}
% ! Other packages
\usepackage{xcolor}
\usepackage{comment}
% * Top matter
\begin{document}
\title
[Response]
{Response}
\maketitle
% * Document body
We are grateful to the referees' for their helpful comments.
We have incorporated all the comments made by the referees.
Here are the specific changes in response to the comments of the referees.
\section*{Response to the comments of Referee 1}
\begin{enumerate}
\item
  p.4 l.4: It has to be They showd that. 

  {\it Changed as suggested.}
\item
  p.12 l.-10: Equation (2.4) is not only integrating both sides of (2.3), please explain more.
  
  {\it We added some explanations before (2.4).}
\item
  p.20 l.4: In the first equality in the denominator it has to be $y^{1+\gamma(\zeta_s)}$.
  
  {\it We think the referee means $y^{1+\gamma(\xi_s)}$.
    Changed as suggested. }
\item
  p.20 l.10 eq.(3.11): You don't need to add $x$ in the norm of $\|\kappa \gamma \phi^{\gamma - 1}\|$.
  
  {\it Done, as suggested.}
\item
  p.20 eq.(3.13) and p.21 in 9 more times: You missed an $m$ in $\left\langle v, \phi^* \right\rangle_m$.
  
  {\it We added them back, as suggested.}
\item
  p.21 l.1: I will add: On the other hand, according to (3.2) and Proposition 3.2.
  
  {\it Added, as suggested.}
\item
  p.23 l.-4: I will add then by Assumption 4, the spine representation, Campbell's formula...
  
  {\it Changed, as suggested.}
\item
  p.25 eq.(3.27) and the next equation: I think that in your equations you missed a factorial, i.e.
  \begin{align}
    F_\alpha(\theta) 
    \leq \frac{C^k}{k!} \left( \rho C+1 \right) \theta^k 
  \end{align}
  and
  \begin{align}
    F_\alpha(\theta) 
    \leq \frac{C^{k+1}}{(k+1)!} \left( \rho  C+1 \right) \theta^{k+1}
  \end{align}
  The reason is that in your next computation, Page 25 line -4 and -3. You missed an $\frac{1}{k+1}$ from the integration. 
  In this case, you need to add a $\frac{1}{k}$ on the last equation of page 25.
    
  {\it We give a new proof of this Lemma which is shorter and better.}
\item
  p.29 in step 3: I will add: Notice that, by (1.20) and since $\left\langle f, \phi^* \right\rangle_m = 1$.
  
  {\it Added, as suggested.}
\end{enumerate}
\section*{Response to the comments of Referee 2}
\begin{enumerate}
\item
  {\bf (Important!)} General remark: There are a number of restrictive looking assumptions and thus it is crucial that the authors provide examples which are covered by the settings.
  It is true that before Assumption 4, there is a reference to [24] `for a list of examples,' but this is somewhat hidden.
  At least a brief summary of examples should be placed at a more visible spot that convinces the reader that the scope is not as restricted as it seems.

  In [24] there are indeed 12 examples given. 
  I would at least vaguely describe (some of) these, possibly in an appendix and definitely without proofs, just mentioning that details and proofs (of these being examples) are in [24].
  If it is relegated to an appendix then somewhere at the beginning at a visible spot I'd refer the reader to the appendix. 
  
  {\it TBD.}
\item
  p.2 in the middle: define positively regular.
  
  {\it We added the definition, as suggested.}
\item
  p.2. l.4: `They showed' instead `he showed'.
  
  {\it Changed, as suggested.}
\item
  p.4. display.4: Explain how this fits into the general formula with the integral term.
  
  {\it We explained it with more details.}
\item
  p.4. (1.16) and (1.17): perhaps mention that the limits do not depend on $x$ resp. $\mu$.
  
  {\it Done, as suggested.}
\item 
  p.5 right after (1.18): Here $\beta$ has not been defined yet, only later in Assumption 4 it will be.
  
  {\it We added a new sentence there which gives the function $\beta$.}
\item
  p.6 l.-15: Why isn't it $P_t^\beta \phi(x) = e^{\lambda t}\phi(x)$ and $P_t^{\beta*} \phi^*(x) = e^{\lambda t}\phi^*(x)$?
  
  {\it Should be $P_t^\beta \phi(x) = e^{\lambda t}\phi(x)$ and $P_t^{\beta*} \phi^*(x) = e^{\lambda t}\phi^*(x)$. Corrected as suggested.}
\item
  p.6 paragraph after Assumption 2: The notation is a bit confusing. So $\phi$ is the principal eigenfunction for the motion with the potential $\beta$ but $\varphi$ is the same without the potential. 
  This should be made more clear.
  
  {\it We added a paragraph after Assumption 2 to make this more clear. 
    Note that in the new version the notation $\varphi$ and $\varphi^*$ is changed into $\widetilde \phi$ and $\widetilde \phi^*$, respectively.}
\item
  Same place ({\bf Important!}): It is not clear why $\varphi$ is well defined. 
  Let's say $\xi$ is a diffusion process.
  If $\mathcal L$ is the elliptic operator corresponding to it and $\lambda(\mathcal L)$ is its (generalized) principal eigenvalue then $\mathcal L - \lambda(\mathcal L)$ may or may not have a Green's function.
  In the second case everything is fine and $\varphi$ is uniquely determined up to constant multiples; it is the ground state.
  However, in the first case the cone of positive harmonic functions w.r.t. $\mathcal L - \lambda(\mathcal L)$ is not necessarily one dimensional. 
  For example, if we know that $\xi$ is recurrent (and so $\lambda(\mathcal L) = 0$), then we are fine.
  But when it is transient, uniqueness does not follow.
  
  {\it Assumption 2 excluded the first case. We added a paragraph after Assumption 2 which explains that the principal eigenfunction without the potential is indeed well defined. 
    Note that in the new version the notations $\varphi$ and $\varphi^*$ are changed into $\widetilde \phi$ and $\widetilde \phi^*$, respectively.}
\item
  p.8 l.6: I would rather say that the family of functions is just $(e^{\lambda t} \phi)_{t\geq 0}$, while $(e^{\lambda t} X_t(\phi))_{t\geq 0}$ is indeed the martingale.
  
  {\it We restated this paragraph to make our intuitive explanations of the spine decomposition theorem more accurate. }
\item
  p.11 Lemma 2.4 and Corollary 2.5:
  Why do we suddenly use $\alpha$ instead of $\gamma$?
  
  {\it Should always use $\alpha$.
     We changed several $\gamma$ into $\alpha$ there.} 
\item
  p.12: Explain how you get (2.4) from (2.3).
  
  {\it We explained this with more details.}
\item
  p.13 top: Why not `$\mathcal N$ is a Poisson random measure on $\mathbb W$'?
  
  {\it Changed as suggested.}
\item
  p.14 top: Give an intuitive description of what the three bullet points formulate.
  
  {\it We gave an intuitive explanations in the paragraph before those bullet points. }
\item
  p.17 l.-7: The RHS has a factor $e^{\lambda t}$ too. 
  Also, you should explain that this follows from the ergodicity of the $\phi$-transformed motion.
  (Two lines later you should refer to Proposition 3.1.)
  
  {\it We added more details there as suggested.}
\item
  p.18 l.8: ... with the same speed.
  
  {\it Changed, as suggested.}
\item
  p.18 l.-11: Please remind us that $\nu = \phi^* m$.
  
  {\it Done, as suggested.}
\item
  p.19 above (3.10): On the other hand,...
  
  {\it Done, as suggested.}
\item 
  p.19-20: It's easier to read the proof backwards. So perhaps you should tell at the beginning that your goal is to show (3.12) and (3.13), etc.
  For the same reason, I'd move the few lines ``Proof of Thm 1.1(2)'' from p.22 to the beginning so that one can understand why the two propositions are needed.
  
  {\it We rearranged the Proof of Thm 1.1 (2) so that it is easier for the readers to get the structure of the Proof. }
\item
  p.21 l.1: Why is this ``according to (3.2)''?
  
  {\it We explained this in more details.}
\item
  p.21 before (3.14): Make it one sentence: if ... then ...
  
  {\it Done, as suggested.}
\item
  p.21 (3.14): $m$ is missing from the scalar product.
  
  {\it We added it back, as suggested.}
\item
  p.24 Lemma 3.5: Say that the assumption on $\gamma_0$ is $\gamma_0 \in (1,2)$. 
  This looks like a stand alone lemma, so perhaps the reader does not connect it with the previous definition of $\gamma_0$.
  
  {\it It is indeed a stand along lemma. 
    We changed all $\gamma_0$ into $\alpha$ in this Lemma.
    Similar changes are made in the next Lemma due to same reasons.}
\item
  p.24 l.-2: When $0 < p < 1$, this is in fact not even a norm.
  
  {\it We agree. 
    This doesn't compromise the later proofs.
    We emphasized this in the new version to avoid any confusions.}
\item
  p.26 l.1: explain a bit more on ``Since this is true for all $k$, we get that $F(\alpha+\theta) = 0$ for all $\theta \leq 1/C, \alpha \leq \rho$, and so $F = 0$ on $[0, \rho+1/C]$.
  
  {\it The prove of this Lemma is simplified and the concerned argument is deleted.}
\item
  pp.24-26: Don't you what to move the analytical lemmas Lemma 3.5 and Lemma 3.6 move into an appendix?
  
  {\it Those Lemmas are now moved into an appendix as suggested.}
\item
  p.26: Instead of (3.28), why don't you simply say that $F$ satisfies the inequality in Lemma 3.6 (number that one) with $C = \gamma_0^{1/(\gamma_0 - 1)}$?
  
  {\it We should simply say that. 
    Done, as suggested.}
\item
  p.28 end of Step 1: $J_g(t,r,\xi)$ and the two other terms you upper estimate are random variables.
  So, do you mean the upper bounds almost surely?
  
  {\it Yes, we mean almost surely. 
  We added ``almost surely'' at several places as suggested.}
\item
  p.29 l.-12: You should also recall that $\left\langle f, \phi^* \right\rangle_m = 1$.
  
  {\it Done, as suggested.}
\item
  p.32 l.-7: reverse instead of revers. 
  Also ``use this ... to have that ...''.

  {\it Done, as suggested.}
\item
  p.33 last sentence: I would say: ``Finally, $M \equiv 0$ clearly implies that $\lim_{t\to \infty} I_3(t,\theta,x) = 0$, and thus completes the verification of (3.29).''

  {\it Done, as suggested.}
\end{enumerate}
% * Bibliographic references
\bibliographystyle{plain}
\bibliography{../orggtd/bib.bib}
\end{document}